%% Generated by Sphinx.
\def\sphinxdocclass{report}
\documentclass[letterpaper,10pt,english]{sphinxmanual}
\ifdefined\pdfpxdimen
   \let\sphinxpxdimen\pdfpxdimen\else\newdimen\sphinxpxdimen
\fi \sphinxpxdimen=.75bp\relax
\ifdefined\pdfimageresolution
    \pdfimageresolution= \numexpr \dimexpr1in\relax/\sphinxpxdimen\relax
\fi
%% let collapsible pdf bookmarks panel have high depth per default
\PassOptionsToPackage{bookmarksdepth=5}{hyperref}

\PassOptionsToPackage{warn}{textcomp}
\usepackage[utf8]{inputenc}
\ifdefined\DeclareUnicodeCharacter
% support both utf8 and utf8x syntaxes
  \ifdefined\DeclareUnicodeCharacterAsOptional
    \def\sphinxDUC#1{\DeclareUnicodeCharacter{"#1}}
  \else
    \let\sphinxDUC\DeclareUnicodeCharacter
  \fi
  \sphinxDUC{00A0}{\nobreakspace}
  \sphinxDUC{2500}{\sphinxunichar{2500}}
  \sphinxDUC{2502}{\sphinxunichar{2502}}
  \sphinxDUC{2514}{\sphinxunichar{2514}}
  \sphinxDUC{251C}{\sphinxunichar{251C}}
  \sphinxDUC{2572}{\textbackslash}
\fi
\usepackage{cmap}
\usepackage[T1]{fontenc}
\usepackage{amsmath,amssymb,amstext}
\usepackage{babel}



\usepackage{tgtermes}
\usepackage{tgheros}
\renewcommand{\ttdefault}{txtt}



\usepackage[Bjarne]{fncychap}
\usepackage{sphinx}

\fvset{fontsize=auto}
\usepackage{geometry}


% Include hyperref last.
\usepackage{hyperref}
% Fix anchor placement for figures with captions.
\usepackage{hypcap}% it must be loaded after hyperref.
% Set up styles of URL: it should be placed after hyperref.
\urlstyle{same}

\addto\captionsenglish{\renewcommand{\contentsname}{Basics:}}

\usepackage{sphinxmessages}
\setcounter{tocdepth}{1}



\title{THAT docu}
\date{Feb 07, 2022}
\release{}
\author{anabrid}
\newcommand{\sphinxlogo}{\vbox{}}
\renewcommand{\releasename}{}
\makeindex
\begin{document}

\pagestyle{empty}
\sphinxmaketitle
\pagestyle{plain}
\sphinxtableofcontents
\pagestyle{normal}
\phantomsection\label{\detokenize{index::doc}}


\sphinxAtStartPar
Welcome to the THAT website, a homepage generated and maintained by anabrid GmbH.

\noindent{\hspace*{\fill}\sphinxincludegraphics[width=400\sphinxpxdimen]{{THAT_front01}.jpg}}

\sphinxAtStartPar
This Wiki contains educational articles for basic understanding of electronic analog computers and documentation about \sphinxstylestrong{The Analog Thing} (or THAT for short), an affordable Open Source Analog Computer with a focus on education, hobbyists, programmers, scientists and everybody interested in non\sphinxhyphen{}traditional computing architectures. The THAT can be purchased from anabrid or assembled/ recreated by anybody with basic experience in electronics and soldering.


\chapter{\sphinxstylestrong{Basics}}
\label{\detokenize{index:basics}}

\section{Differential equations}
\label{\detokenize{rst/basics_differential_equations:differential-equations}}\label{\detokenize{rst/basics_differential_equations::doc}}

\section{Machine Unit}
\label{\detokenize{rst/basics_machine_unit:machine-unit}}\label{\detokenize{rst/basics_machine_unit::doc}}
\sphinxAtStartPar
The \sphinxstylestrong{THAT}, as most other electrical analog computers, can compute values in the range from \sphinxhyphen{}1 to +1. So every program needs to be checked if it exceeds +/\sphinxhyphen{}1 at any point. This check can also happen theoretically as well as experimentally. If a running programm exceeds +/\sphinxhyphen{}1, most analog computers will change into an overload state to protect the computing elemtents so no damage is to be expected. However the computing results are useless in case of overload.

\sphinxAtStartPar
The THAT indicates overload state with a red LED light with \sphinxstylestrong{OL} marked.

\sphinxAtStartPar
The \sphinxstylestrong{machine unit} is a technical variable and \sphinxstylestrong{represents the maximum voltage} which can be applied to the computing elements of an electrial analog computer.

\sphinxAtStartPar
The machine unit is system specific, the \sphinxstylestrong{THAT has a machine unit of +/\sphinxhyphen{} 10 volts.}

\sphinxAtStartPar
The CHINCH\sphinxhyphen{}outputs (x, y , z, u) have a 10 to 1 voltage devider so the results given by those outputs can only give values between \sphinxhyphen{}1 and +1.

\newpage


\section{Computing elements}
\label{\detokenize{rst/basics_computing_elements:computing-elements}}\label{\detokenize{rst/basics_computing_elements::doc}}

\subsection{Coefficient potentiometer}
\label{\detokenize{rst/computing_elements/potentiometer:coefficient-potentiometer}}\label{\detokenize{rst/computing_elements/potentiometer::doc}}

\begin{savenotes}\sphinxattablestart
\centering
\begin{tabular}[t]{|\X{75}{225}|\X{75}{225}|\X{75}{225}|}
\hline

\noindent{\hspace*{\fill}\sphinxincludegraphics[width=150\sphinxpxdimen]{{potentiometersymbol}.png}\hspace*{\fill}}
&
\noindent{\hspace*{\fill}\sphinxincludegraphics[width=200\sphinxpxdimen]{{THATv1.0_potti_single}.jpg}\hspace*{\fill}}
&
\noindent{\hspace*{\fill}\sphinxincludegraphics[width=100\sphinxpxdimen]{{THATv1.0_potti_knob}.jpg}\hspace*{\fill}}
\\
\hline
\end{tabular}
\par
\sphinxattableend\end{savenotes}

\sphinxAtStartPar
The simplest and only passive standard component ist the \sphinxstylestrong{coefficient petentiometer}.

\sphinxAtStartPar
Commonly realized as precision potenetiometer connected as voltage divider, it is used to multiply a given input with a value between 0 and 1, so \sphinxstylestrong{a*x} is generated.

\sphinxAtStartPar
It can also be used as dividing factor so \sphinxstylestrong{x/a} is generated.

\sphinxAtStartPar
\sphinxstylestrong{Round/circular panel \textendash{}\textgreater{} Input}

\sphinxAtStartPar
\sphinxstylestrong{Triangular panel \textendash{}\textgreater{} Output}


\subsection{Operation amplifier}
\label{\detokenize{rst/computing_elements/operationamp:operation-amplifier}}\label{\detokenize{rst/computing_elements/operationamp::doc}}
\sphinxAtStartPar
The \sphinxstylestrong{operation amplifier} is the most important electronic part in electrical analog computers.

\sphinxAtStartPar
The term \sphinxstyleemphasis{operation amplifier} was introduced and defined by \sphinxstyleemphasis{John Ragazzini} in 1947:
\begin{quote}

\sphinxAtStartPar
“As an amplifier so connected can perform the mathematical operations of arithmetic and calculus on the voltages aookued ti its input, it is hereafter termed an \sphinxstyleemphasis{Operational Amplifier}.”
\end{quote}

\sphinxAtStartPar
In it’s simplest form, the operation amplifier can be found in analog computers as so called \sphinxstyleemphasis{open amplifiers}, where the positive input is eleictrically grounded while it’s negitave input is connected to a resistor. This is used to generate inverse functions (e.g. a multiplier in combination of an open amplifier functions as divider).


\subsection{Summer}
\label{\detokenize{rst/computing_elements/summer:summer}}\label{\detokenize{rst/computing_elements/summer::doc}}

\subsection{Integrator}
\label{\detokenize{rst/computing_elements/integrator:integrator}}\label{\detokenize{rst/computing_elements/integrator::doc}}

\subsubsection{Priciple circuit}
\label{\detokenize{rst/computing_elements/integrator:priciple-circuit}}

\subsubsection{Practical circuit}
\label{\detokenize{rst/computing_elements/integrator:practical-circuit}}

\subsection{Function generators}
\label{\detokenize{rst/computing_elements/function_generators:function-generators}}\label{\detokenize{rst/computing_elements/function_generators::doc}}

\subsection{Creating inverse functions}
\label{\detokenize{rst/computing_elements/inverse_functions:creating-inverse-functions}}\label{\detokenize{rst/computing_elements/inverse_functions::doc}}

\subsection{Multiplier}
\label{\detokenize{rst/computing_elements/multiplier:multiplier}}\label{\detokenize{rst/computing_elements/multiplier::doc}}

\subsubsection{Devision and root}
\label{\detokenize{rst/computing_elements/multiplier:devision-and-root}}

\subsection{Comperators}
\label{\detokenize{rst/computing_elements/comperators:comperators}}\label{\detokenize{rst/computing_elements/comperators::doc}}

\subsection{Coordinate converter}
\label{\detokenize{rst/computing_elements/coordinate_converter:coordinate-converter}}\label{\detokenize{rst/computing_elements/coordinate_converter::doc}}

\subsection{Delays}
\label{\detokenize{rst/computing_elements/delays:delays}}\label{\detokenize{rst/computing_elements/delays::doc}}

\subsection{Noise generators}
\label{\detokenize{rst/computing_elements/noise_generators:noise-generators}}\label{\detokenize{rst/computing_elements/noise_generators::doc}}

\subsection{Patch field}
\label{\detokenize{rst/computing_elements/patch_field:patch-field}}\label{\detokenize{rst/computing_elements/patch_field::doc}}

\subsection{Output devices}
\label{\detokenize{rst/computing_elements/output_devices:output-devices}}\label{\detokenize{rst/computing_elements/output_devices::doc}}

\section{Operation modes of analog computers}
\label{\detokenize{rst/basics_operation_modes:operation-modes-of-analog-computers}}\label{\detokenize{rst/basics_operation_modes::doc}}

\section{Programming analog computers}
\label{\detokenize{rst/basics_programming:programming-analog-computers}}\label{\detokenize{rst/basics_programming::doc}}

\section{Visualization of computing results}
\label{\detokenize{rst/basics_visualization:visualization-of-computing-results}}\label{\detokenize{rst/basics_visualization::doc}}

\chapter{\sphinxstylestrong{Applications}}
\label{\detokenize{index:applications}}

\section{Hindmarsh\sphinxhyphen{}Rose neuron model}
\label{\detokenize{rst/applications/hindmarsh-rose_neuron:hindmarsh-rose-neuron-model}}\label{\detokenize{rst/applications/hindmarsh-rose_neuron::doc}}\label{\detokenize{rst/applications/hindmarsh-rose_neuron::doc}}

\section{Lorenz attractor}
\label{\detokenize{rst/applications/lorenz_attractor:lorenz-attractor}}\label{\detokenize{rst/applications/lorenz_attractor::doc}}

\chapter{\sphinxstylestrong{History}}
\label{\detokenize{index:history}}

\section{Early electronic analog computers:}
\label{\detokenize{rst/history_first_electronic_analog_computers:early-electronic-analog-computers}}\label{\detokenize{rst/history_first_electronic_analog_computers::doc}}

\section{Mechanical analog computers}
\label{\detokenize{rst/history_mechanical_analog_computers:mechanical-analog-computers}}\label{\detokenize{rst/history_mechanical_analog_computers::doc}}

\section{Electronic analog computers since 1950}
\label{\detokenize{rst/history_system_examples_1950-now:electronic-analog-computers-since-1950}}\label{\detokenize{rst/history_system_examples_1950-now::doc}}


\renewcommand{\indexname}{Index}
\printindex
\end{document}